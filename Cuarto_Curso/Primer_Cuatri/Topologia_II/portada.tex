\documentclass[12pt,a4paper]{article}
\usepackage{graphicx} % Para incluir imágenes
\usepackage{geometry} % Para ajustar márgenes
\geometry{top=3cm,bottom=3cm,left=3cm,right=3cm}
\usepackage{amsthm} % Paquete recomendado para entornos matemáticos

\newtheorem{theorem}{Teorema}[section] % Teoremas numerados por sección
\newtheorem{prop}{Proposición}[section] % Proposiciones numeradas por sección
\newtheorem{lema}{Lema}[section] % Lemas numerados por sección
\newtheorem{definition}{Definicion}
\newtheorem{corollary}{Corolario}[section]
\newtheorem{example}{Ejemplo}[section]
\newtheorem{notation}{Notación}[section]


\begin{document}

% Portada
\begin{titlepage}
    \centering

    {\Huge \textbf{Teorema de Seifert-vanKampen}} \\[1.5cm]
    \includegraphics[width=0.6\textwidth]{/home/el_dramas/Desktop/ugr.jpg} \\[1.5cm] % Imagen
    {\Large Autor: Lucas Hidalgo Herrera } \\[0.5cm] % Autor
    {\large Grado: Doble Grado en Ingeniería Informática y Matemáticas} \\[0.5cm]
    {\large Asignatura: Topología II} \\[0.5cm]% Curso o asignatura
    {\large Fecha: \today} % Fecha actual

    \vfill

\end{titlepage}
\newpage
\renewcommand{\contentsname}{Índice General}
\tableofcontents
\newpage

\end{document}
